\documentclass[10pt,conference,compsocconf]{IEEEtran}

\usepackage{hyperref}
\usepackage{graphicx}
\usepackage{xcolor}
\usepackage{blindtext, amsmath, comment, subfig, epsfig }
\usepackage{grffile}
\usepackage{caption}
%\usepackage{subcaption}
\usepackage{algorithmic}
\usepackage[utf8]{inputenc}


\title{CS-523 SecretStroll Report}
\author{Author 1, Author 2}
\date{April 2020}

\begin{document}

\maketitle

\begin{abstract}
    SecretStroll is a privacy preserving LBS. SecretStroll is build on top of 3 core modules: an Anonymous
    \textit{Attribute-based Credential protocol} for authorization, <part 2> and <part 3>.
\end{abstract}

\section{Introduction}

Provide a brief introduction about the aim of the project, and your road-map about the design/implementation for each sub-part.
SecretStroll is a Location Based System (LBS), which aims to provide users with information about Points of Interest (POI)
near their location, satisfying user-specified criteria (i.e retrieve POIs near user location which correspond to the description
of restaurant).\newline
LBS are prone to be privacy-sensitive system: if not implemented following a \textit{Privacy by Design} paradigm, they can leak
many information about users, such as their location and movement patterns, which can lead to more privacy-disruptive disclosure attacks.
For this reason, we designed SecretStroll by identifying layers at which the system could potentially leak information and implementing
privacy-preserving mechanism at each layer.\newline
SecretStroll is formed by 3 core layers: an \textit{Attribute-based Credential} protocol, <part2> and <part3>.
\subsection{Attribute-based credential}
SecretStroll exploits an Attribute-based credential Protocol to authorize users to fetch POIs of a given type they subscribed for.
At issuing time, users will provide their username together to a set subscriptions (i.e types of POIs) which are currently supported
by SecretStroll system. SecretStroll server will verify the validity of user's request and generate a valid credential for the specified
subscriptions and username. Users will later show their credentials togheter with a subset of their subscriptions: in this step,
the \textit{showing} protocol, users will have to prove to the server to have a valid credential over their attributes
(including, but not limited to, their subscriptions). Users will also use their credential to perform a digital signature
on their current location. If the proof suceeds, the server will reply with a list of POIs matching the types requested by the user.
\subsection{part 2}
\subsection{part 3}
\section{Attribute-based credential}
Explain how you mapped the system to the attribute based credential. How did you
use the Fiat-Shamir heuristic?
For the implementation of the authorization through \textit{Attribute-based credentials}, we have carefully followed the protocol
described by \textit{Pointcheval and Sanders} \cite{PS_signature}. Nevertheless, during the development of the \textit{ABC} system,
some design challenges needed to be solve, and it is worth highlighting their solution:
\begin{itemize}
    \item Server-Client agreement on the attribute domain: as described in \cite{PS_signature}, before even starting the protocol,
    server and client must agree on the public parameters, including the number $L$ of possible attributes. Moreover, users should choose
    attributes (subscriptions) which are recognized by the server. To this end, we decided to embed the list of available subscriptions
    in the server public key (which follows the description of the paper). The 'attribute domain' is thus formed by the $L-2$ possible
    subscriptions, followed by the username and, following a common approach in designing \textit{ABC} protocols, a client secret key.
    Clients are expected to choose their subscriptions from the provided set.
    \item Attribute encoding: once we designed the server-client agreement in such a way that each parameter of the public key
    (namely the $Y_i$ and $Y^{~}_{i}$) was mapped to one attribute, we decided to encode attributes in the followng way:
    if the user decides to subscribe to service $i$, then the exponent of parameter $i$ of server public key(i.e the attribute),
     will be a fixed prime number in $Z_p$ (where $p$ is the order of the prime groups defined in the paper),
    to which we refer as \texttt{SUBSCRIBED_YES}. Conversely, any service to which the client does not subscribe,
    will be encoded with a different fixed prime, \texttt{SUBSCRIBED_NO}. The 'username' attribute will be encoded through
    \texttt{int.from_bytes(username.encode(), 'big')} method available in \textit{python}. Client secret key will be a
    random number in $Z_p$.
    \item Fiat-Shamir Heuristic for Issuance Request: the \textit{Issuance Protocol} once again follow thoroughly the paper description.
    Client will choose their \textit{user-defined} attributes, which in our implementation will simply be its secret key,
    togheter with a blinding factor $t$. User commitment $C$ will thus be $g^t.Y_{client-sk}^{client-sk}$. Client will also produce a
    \textit{Non Interactive Zero Knowledge Proof of Knowledge} of his commitment. In order to generate the challenge in a
    non interactive way, we applied \textit{Fiat-Shamir Heuristic} to the \textit{sigma protocol} defined for the
    \textit{Pedersen's Commitment PK}: the provided challenge was the \textit{sha256} digest of the public parameters
\end{itemize}

\subsection{Test}
How did you test the system?
You need to test the correct path and at least two failure paths.

\subsection{Evaluation}
Evaluate your ABC: report communication and computation stats (mean and standard
deviation). Report statistic on key generation, issuance, signing, and
verification.

\section{(De)Anonymization of User Trajectories}

\subsection{Privacy Evaluation}
Provide a privacy analysis of the dataset. You should explicitly state your assumptions, adversary
models, methods, and findings.

\subsection{Defences}
Propose a defence that users of the service could deploy to protect their privacy.  You
should state your assumptions, adversary models, and provide an experimental evaluation of your
defences using the datasets and the grid specification. You should also discuss the
privacy-utility trade-offs of your defence.

\section{Cell Fingerprinting via Network Traffic Analysis}

\subsection{Implementation details}
Provide a description of your implementation here. You should provide details on your data collection methods, feature extraction, and classifier training.

\subsection{Evaluation}
Provide an evaluation of your classifier here -- the metrics after 10-fold cross validation.

\subsection{Discussion and Countermeasures}
Comment on your findings here. How well did your classifier perform? What factors could influence its performance? Are there countermeasures against this kind of attack?

\bibliographystyle{IEEEtran}
\bibliography{bib}
\end{document}
